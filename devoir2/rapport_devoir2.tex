\documentclass[11pt,letterpaper,sans]{article}   % possible options include font size ('10pt', '11pt' and '12pt'), paper size ('a4paper', 'letterpaper', 'a5paper', 'legalpaper', 'executivepaper' and 'landscape') and font family ('sans' and 'roman')

\usepackage[francais]{babel}
\usepackage[utf8]{inputenc}
\usepackage[T1]{fontenc}
\usepackage[margin=2cm]{geometry}
\usepackage{fancyheadings}
\usepackage{listings}


\pagestyle{fancy}
\linespread{2}

\chead{Devoir 2 - Séquentiel, C++ Threads, OpenCl}
\cfoot{Clément Zotti 11 074 801, Ferhat Abbas 10 205 137}

\begin{document}
Nous utilisons la même fonction pour le parallèle et le séquentiel, elle prend un interval en paramètre et fait sont travail.

Le programme mono-para exécute la partie parallèle en premier puis la partie série.
\begin{itemize}
\item C++ sequentiel, parallèle
\begin{lstlisting}
./mono-para aabaabb
aabaabb, 45 4294967268 4294967267 15 19 39 18 
Parallel : 
Time 1.20073 s.
Trouve aabaabb, 45 4294967268 4294967267 15 19 39 18 
Sequentiel : 
Time 1.11264 s.
Trouve aabaabb, 45 4294967268 4294967267 15 19 39 18 
Done
\end{lstlisting}

\item OpenCL
\begin{lstlisting}
Entrez la chaine de caractere de longueur 7
aabaabb
Wait while processing...
Time 10.1712 s.
Le mot de pass trouve est : aabaabb
\end{lstlisting}

\end{itemize}

Le programme parallèle est divisé en deux threads, nous avions 4 coeurs sur la machine de test donc le plus proche diviseur de 26 en entier est 2.
On obtient donc des résultats similaire en c++ séquentiel et parallèle pour l'interval suivant :
aaaaaaa - nnnnnnn

Car, la séparation de l'alphabet se fait comme suit:
aaaaaaa - nnnnnnn
nnnnnnn - zzzzzzz

Donc pour tout ce qui est au dessus de nnnnnnn la partie parallèle est plus rapide d'un facteur d'environ 2 dans notre cas, sinon la partie séquentielle est plus rapide d'un facteur [1.0786, 2.7852] plus le mot de passe est proche de aaaaaaa plus la partie séuqntielle à de meilleur résultats.

La partie OpenCl se calcule de la même facon mais divisé en 26 parties.

\begin{itemize}
\item C++ sequentiel, parallèle
\begin{lstlisting}
./mono-para aazbbbb
nnnonnn, 45 4294967268 4294967267 15 19 39 18 
Parallel : 
Time 33.6309 s.
Trouve nnnonnn, 45 4294967268 4294967267 15 19 39 18 
Sequentiel : 
Time 32.3821 s.
Trouve nnnonnn, 45 4294967268 4294967267 15 19 39 18 
Done
\end{lstlisting}

\item OpenCL
\begin{lstlisting}
Entrez la chaine de caractere de longueur 7
aazbbbb
Wait while processing...
Time 140.748 s.
Le mot de pass trouve est : aabaabb
\end{lstlisting}
\end{itemize}



\begin{itemize}
\item C++ sequentiel, parallèle
\begin{lstlisting}
./mono-para nnnonnn
nnnonnn, 45 4294967268 4294967267 15 19 39 18 
Parallel : 
Time  s.
Trouve nnnonnn, 45 4294967268 4294967267 15 19 39 18 
Sequentiel : 
Time xxx.xxxx s.
%Trouve nnnonnn, 45 4294967268 4294967267 15 19 39 18 
%Done
Trop long pas eu le temps de finir
\end{lstlisting}

\item OpenCL
\begin{lstlisting}
Entrez la chaine de caractere de longueur 7
nnnonnn
Wait while processing...
Time 3.77138 s.
Le mot de pass trouve est : nnnonnn
\end{lstlisting}
\end{itemize}

Il y a quelques mot de passe qui on le même hash ce qui peux causer des problèmes dans la résolution.
\begin{itemize}
\item qqqrrrt = aaabbbd
\end{itemize}


\end{document}


